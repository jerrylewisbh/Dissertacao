\chapter{\hspace*{3pt} Introduction}
\label{chapter:intro}
\section{\hspace*{3pt}Contextualization}

The organization of information has been a concern of human beings since the beginning of the first civilizations, about 4,000 years ago ~\cite{baeza1999modern}. At that time, accounting records, government directives, contracts, and court sentences were kept and organized into clay tablets%citation -- Nurmikko-Fuller, 2018
. Over the years, these tablets have been replaced by paper, and the number of documents increased considerably. Hence the activity of locating them had become a significant challenge for the organization of information.

An example of this is in the classification of books in a library. Librarians ordinarily use classification systems %such as MARC 21 (citation?), and the Dewey Decimal (citation? cited below)
to organize the books on their shelves, clustering together those that are on the same topic. The topics themselves are usually divided into increasingly specific subcategories, forming a hierarchical classification. 

In the early 1990s, the Web appeared. It represented a distributed hypermedia system that enabled users to search for information in a wide variety of areas of knowledge. The large volume of Web documents and the inability to perform extensive editorial control in this system have contributed to the emerging importance of the organization of Web documents, a substantial challenge facing  \gls{ir}, a research area that deals with the problem of representing, organizing and storing information for the user to access them using the computer~\cite{baeza1999modern}. Web directories such as the \gls{odp}, Yahoo! Directory and Google Directory are applications that try to organize Web documents into a hierarchy of topics to make it easier to navigate and retrieve them. 

The expansion and maintenance of these directories have been done manually by publishers %such as?
who analyze the content of Web documents and classify them on particular topics. These manual classification were however ineffective, due mainly to the number of documents published on the Web, and all of them have been discontinued at some point.

If on the one hand the expansion of the Web has represented a challenge for experts aiming to classify the raising number of documents, then on the other, it has brought to the fore a significant social impact by enabling users to participate in the construction and organization of information. Folksonomies, for instance, are collaborative attempts to categorize items of some type, with the aim of helping users in their searches \cite{peters2009folksonomies}.

In this context of the collaborative construction of knowledge on the Web, Wikipedia is the most substantial encyclopedia freely available. It has been developed and curated by a large number of users over a period of time, and contains information covering a very broad range of topics currently found on the Web. Wikipedia is itself organized with a folksonomy, one that takes the form of a category hierarchy: to each Wikipedia article there are one or more categories, which are themselves structured as a collaborative hierarchical structure. Therefore, Wikipedia can be considered as a knowledge graph with an explicit, human-authored form of classification.

As in many classification methods, such as Dewey Decimal Classification\cite{mitchell1996dewey} and Library of Congress Classification\cite{chan2016guide}, an article in Wikipedia can belong to one or more top categories which in some sense represent the topics it covers. Within Wikipedia, the primary purpose of this classification is to facilitate the search for relevant information.


The Wikipedia Categorization scheme is a thesaurus collaboratively constructed and used for indexing the content of Wikipedia pages \cite{voss2006collaborative}. Hence, it can be said that it represents a common or shared understanding of Wikipedia's content. It is a classification made by the user community, rather than one elicited from experts for laypersons. The richness of this type of information, which enables several tasks performed by users on the Web (such as search, information retrieval, recommendation, and clusterization), is also noteworthy.


Categorization plays a crucial role in the future of information search services, and many positive categorization approaches involve the integration of knowledge from Wikipedia. This explains the widespread use of Wikipedia’s article contents and category hierarchy to generate semantic resources that enhance performance on text classification and keyword extraction among other applications \cite{gantner2009automatic}. 

In this context, the primary purpose of this thesis is to create a general-purpose classification method based on the Wikipedia Categorization scheme to categorize text-based content on the Web, for instance, scientific articles, web pages or even posts on social media. Although the API is versatile enough to generate a classification for any textual input, the validity of the method in the scope of this thesis was tested only in a small extent, with texts extracted from question and answer communities.
The method relies on feature extraction from the collective knowledge of Wikipedia contributors, rather than on a traditional classification system created by domain experts. That is, regular users are more likely to successfully access and retrieve information from a Web created by people for people, than from one created solely by experts.

At its current stage, the proposed method can classify content in English and can be accessed at \url{http:\\\\www.TagTheWeb.com.br}.

The Wikipedia categorization scheme was chosen for four main reasons: 

\begin{itemize}
\item  Wikipedia is the largest online encyclopedia and is constructed and maintained by contributors from all over the world.
\item Wikipedia contains an categorization scheme, curated by humans, where all articles are placed within categories that describe their content, and these categories are semantically related to other categories in a rich and meaningful network.
\item Wikipedia categories are words or phrases in natural language, making them easy for regular users of information retrieval systems to understand and interact with.
\item The content on Wikipedia is dynamic, allowing an adaptive and evolving classification method. Unlike other encyclopedias, Wikipedia is often updated in real time.
\end{itemize}


\section{\hspace*{3pt}Motivation}

With the ubiquitous Internet and the rapid growth of the Web, accessing the vast amount of digital text remains a challenge for users. One of the potential solutions is to use automated classification and categorization of Web information \cite{Gabrilovich:2005}.

The amount of data available in digital format on the worldwide Web has increased steadily. According to estimates made in 2014, from 2013 to 2020 the digital universe will increase from 4.4 trillion gigabytes to 44 trillion gigabytes ~\cite{turner2014digital}. Much of the data in the digital universe is in textual formats, such as emails, reports, newsletters, articles, and Web page content. Also, with the advent of the Web 2.0, textual data has been used as a means to disseminate information, whether by postings on social networks, wikis or blogs ~\cite{fuchs2013internet} ~\cite{o2009web}.

Due to the significant amount of textual information produced and made available today, it becomes humanly impossible to organize, analyze, and extract knowledge embedded in textual information. Consequently, mechanisms to accomplish such tasks as removing or at least reducing the need for human intervention have gained importance in recent decades ~\cite{feldman2007text} ~\cite{berry2010text} ~\cite{aggarwal2012mining}. 

The possibility of investigating a method for automatic text classification, capable of assigning categories that are understandable to humans and take into account the collective knowledge encoded into Wikipedia rather than an expert's effort, is what motivates the development of this work. Text classification allows users to find desired information faster by searching relevant categories only (rather than the entire information space), and helps users to develop conceptual views of digital documents. Since the information exists in unstructured form, categorization can allow users to make the most use of texts. 



\section{\hspace*{3pt}Problem Statement}


According to Sebastiani \cite{Sebastiani:2002}, although the first efforts to automate the classification of digital documents were made in the 1960s, a semiautomatic technique based on knowledge engineering was used for document classification until the 1980s.

Whilst a semiautomatic approach can be precise, it has a significant limitation concerning the acquisition of knowledge for the construction of the classifier. This limitation is primarily the need to have at least two human experts involved in the process: a domain expert, with the ability to classify documents in the predefined set of classes; and a knowledge engineer, able to encode the classification in a programming language as a set of rules. This approach is inflexible because each and any iterative stage of changes or new developments in the classification system necessitates the involvement of the two experts to adjust the rules and the classifier. 

\section{\hspace*{3pt}Goals of this Thesis}

The primary goal of this research is to study the viability of taking advantage of this collective body of knowledge to automatically categorize web-based content according to Wikipedia contributors. The primary objective can be broken down into the following specific goals:


\begin{enumerate}[i]

\item Perform a graph-theoretic analysis of the Wikipedia Category Graph, to describe the topology of this structure and to identify the challenges and potentials of employing it for extracting features from Web documents;

\item  Propose an approach for the  extraction of features in text-based resources based on the intelligence embedded in the structure of Wikipedia categories



\item Suggest a method for representing documents, which is based on Wikipedia categories, and allows for automatic classification;

\item Design, execute and analyze the results of an experimental study involving crowd-sourcing, to verify whether humans recognize the classification 
generated by the proposed method.


\end{enumerate}

\section{\hspace*{3pt} Project Overview}

Automatic text classification is a process where a category or a set of categories are assigned to a textual resource, based on specific criteria.

There are several methods for performing automatic text categorization. This project focuses on categorizing text based on the named entities found in the text and its relation to a set of predefined categories.

A processing chain to generate a generic categorization consists of three steps:

\begin{enumerate}[(i)]
\item \hyperref[sec:text-annotation]{Text Annotation};

Automatically extract structured information from unstructured text and link it to an external knowledge base in the \gls{lod}. For this thesis, DBpedia was used because it is based on information extracted from Wikipedia.

\item \hyperref[sec:categories-extraction]{Categories Extraction};

In this step, the entity relationships are traversed to find a more general representation of the entity: their categories. All categories associated with the entities identified in the text are extracted and indexed.

\item \hyperref[sec:doc-representation]{Document Representation}.

The set of all Wikipedia categories cannot be directly used as a feature for categorization, because different texts will have a different set of categories, making it impossible to categorize and compare them. To reduce this dimensionality, the applied approach consists of navigating the Category Graph from each category extracted in the previous step towards the top of the graph by all the shortest paths between the category and the main topics. 

Based on the influence of each main topic category on the resource, a document representation of the calculated categorization was generated as a multidimensional vector.

\end{enumerate}

\section{\hspace*{3pt}Main Contributions}

This thesis has two primary categories of outcomes: i) Scientific Contributions and ii) Technical Contributions.

\subsection{\hspace*{3pt}Scientific Contributions}


\begin{itemize}
\item The proposal of a method for the extraction of features and representation of text-based resources, based on the categorization scheme of Wikipedia.

\item The results of the experimental crowd-sourcing study indicating a positive correlation between the classification generated by the proposed method and the understanding of people about the content of the documents evaluated.

\item The results of an updated analysis of the Wikipedia Category Graph, which indicates that like other networks used for natural language processing problems, it is also a small-world and scale-free network.


\end{itemize}


\subsection{\hspace*{3pt}Technical Contributions}

\begin{itemize}
\item TagTheWeb\footnote{\url{http://www.tagtheweb.com.br}}, a public, documented \footnote{\url{http://documenter.getpostman.com/view/1071275/tagtheweb/77bC7K}} and open-source API capable of receiving any textual resource and processing each of the three phases described in the proposed approach (see \ref{sec:approach}).

\item  The Wikipedia Category Graph snapshot from October of 2016 filtered and represented in Neo4J\footnote{\url{http://www.neo4j.com}} and graph-tools\footnote{\url{http://www.graph-tool.skewed.de}}.  

\item A dataset \footnote {\url{http://www.github.com/jerrylewisbh/TagTheWeb}} containing all nodes of the Wikipedia Category Graph and the measures of centrality, in-degree, out-degree, clustering coefficient, and PageRank. An example with the first 50 categories can be seen in Appendix \ref{app:dataset-categories}.

\end{itemize}
Part of this research has already been published: \textit{TagTheWeb: Using Wikipedia Categories to Automatically Categorize Resources on the Web} \cite{medeiros2018tagtheweb}.

\section{\hspace*{3pt}Thesis Outline}

This thesis is organized into eight chapters, this Introduction being the first of them. The other chapters are organized as follows, and describe, respectively:

\begin{itemize}

\item \textbf {Chapter \ref{chapter:wikipedia}:} The main features and the organization of Wikipedia, the primary source of information.


\item \textbf {Chapter \ref{chapter:graph}:} The graph-theoretic analysis carried out on the Wikipedia Category Graph.

\item\textbf {Chapter \ref{chapter:related-concepts}:} The fundamental concepts needed to understand the method described in this thesis.

\item \textbf {Chapter \ref{chapter:methodology}:} The steps of the proposed method illustrated by a running example.

\item \textbf {Chapter \ref{chapter:experiments}:} The evaluation methods employed, and the results of computational and crowd-sourcing experiments.

\item \textbf {Chapter \ref{chapter:related-works}:} The closest related works that served as inspiration for the development of this research.

\item \textbf {Chapter \ref{chapter:conclusion}:} The conclusions extracted from the experiments, the contributions of the research in a general context, its limitations, and perspectives of future work.

\item \textbf{Appendix \ref{app:representing-structre}} details how the information was extracted from Wikipedia and represented as a directed graph.

\item \textbf{Appendix \ref{app:percentagem-distribution}} presents the percentage distribution of categories along ten stack exchanged communities used in the experiment described in section \ref{section:proof-of-concept}.

\item \textbf{Appendix \ref{app:crowd-details}} contains details about the crowd-sourcing experiment described in section \ref{sec:crowd-sourcing-study}.

\item \textbf{Appendix \ref{app:dataset-categories}} Shows a sample of the dataset containing all nodes of the Wikipedia Category Graph and the measures of centrality, indegree, outdegree, clustering coefficient, and PageRank. %\ref{sec:crowd-sourcing-study}.


\end{itemize}
