\chapter{\hspace*{3pt} Related Works}
\label{chapter:related-works}


The knowledge encoded in Wikipedia has been used by many researchers as a tool for performing several tasks, including text categorization \cite{Gabrilovich:2006}, co-reference resolution \cite{strube2006wikirelate}, predicting document topics \cite{schonhofen2009identifying}, automatic word sense disambiguation \cite{mihalcea2007using}, searching synonyms \cite{krizhanovsky2006synonym} and computing semantic relatedness 
\cite{ponzetto2006exploiting}, \cite{gabrilovich2007computing}, \cite{milne2007computing}.


The method proposed in this thesis aims to categorize text-based resources based on the Named Entities found in the text and assert its relation to a set of predefined categories in Wikipedia in a way that users can understand and make use of% the outcomes
. The Wikipedia category structure was represented as a graph and used to determine the categorization based on the shortest paths between the categories associated with the entities and a set of more generic predefined Wikipedia categories. Classifiers can be created in many forms, and to focus on different features. There are a number of existing projects that share the goal of creating text classifiers from a knowledge base:

\begin{itemize}


\item Overcoming the Brittleness Bottleneck using Wikipedia: Enhancing Text Categorization with Encyclopedic Knowledge \cite{Gabrilovich:2006}

\item What's in Wikipedia?: mapping topics and conflict using socially annotated category structure \cite{kittur2009s}

\item Identifying document topics using the Wikipedia category network \cite{schonhofen2009identifying}

\item Bringing Bag-of-phrases to ODP-based Text Classification \cite{7425975}

\item Toward Robust Classification using the Open Directory Project \cite{7058134}

\item A Method for Automated Document Classification
Using Wikipedia-Derived Weighted Keywords \cite{7062484}

\item Classification of Comments by Tree Kernels Using the Hierarchy of Wikipedia for Tree Structures \cite{takeda2016classification}

\item Wikitop: Using Wikipedia Category Network to Generate Topic Trees \cite{kumar2017wikitop}



\end{itemize}


Gabrilovich and Shaul Markovitch \cite{Gabrilovich:2006} described a method for finding the Wikipedia article most similar to a given document, and extends that document's \gls{bow} representation with the words occurring in the Wikipedia article. This approach provides a greater number of topic-specific words to the documents, which makes it easier to classify them with standard text classification techniques. The idea behind this approach is similar to the one outlined in this thesis regarding the attempt to provide richer semantics to the representation of documents than could be achieved using the \gls{bow} approach. However, their approach is very complex and yet fails to add semantic context to the representation.

The method proposed by Kittur and Chi \cite{kittur2009s} is similar to the approach outlined in this thesis. Their goal was to automatically assign a Wikipedia article to a set of what they call macro-categories (a subset of Wikipedia categories that is at the top of the hierarchy). The main difference is that their approach is limited to articles inside Wikipedia, and cannot be generalized for other text-based resources -- this limitation is, however, addressed in the research outlined in this thesis. Kittur and Chi \cite{kittur2009s} evaluated the approach by comparing the attributions made by their method with the judgment of human raters. Although they present a moderate positive correlation between the method and the judgments, the experiment was realized on a small scale by comparison to the one carried out in this thesis. 

A similar approach was applied by Sch\"onhofen \cite{schonhofen2009identifying} to determine whether documents could be categorized by exploring features from Wikipedia. They validated their method first by predicting categories of the Wikipedia articles themselves, and then by classifying documents of an external dataset based on their Wikipedia categories. The main difference between this research and the one reported on in this thesis is that in the latter, the entire category structure has been taken into account, while the approach described in \cite{schonhofen2009identifying} looked exclusively at categories retrieved from the matched Wikipedia article titles (hence, their approach was more limited). Leveraging the hierarchical structure of the categories, as was completed in this thesis, was allocated to future work by Sch\"onhofen.

Ha et al.  \cite{7058134} addressed the problem of sparsity in the Open Directory Project categorization structure by testing several approaches for text classification. They demonstrated that training data expansion is one of the promising directions to deal with the sparse characteristic of the ODP dataset. 
One of the interesting findings of this work is that distance-based weighting had a better result over the other methods tested. The approach was only evaluated in pages manually classified in the ODP. 

Shin at al. \cite{7062484} proposed a method for overcoming the limitations of \gls{bow} by representing the texts with a group of phrases rather than words alone. They employed a syntactic tree to extract phrases from Open Project Directory and applied a phrase selection method to alleviate the high dimensionality problem of bag-of-phrases. Although the approach proposed by them shares the goal of providing more semantic features for document representation in the classification task, it makes use of a knowledge base that has been built by experts, and discontinued in 2017. Hence, the approach outlined in this thesis - which leverages the knowledge contained in Wikipedia - is broader and more flexible to change. 


Biuk-Aghai and Ng \cite {7062484} also presented a method for the automatic classification of scientific articles based on the categories of Wikipedia. They too take advantage of the category graph to generate the classification. The first difference is that while the method utilised in the context of this thesis extracts concepts from text based on the recognition of named entities, \cite {7062484} uses a statistical approach to extract keywords considered relevant. The second difference is that to aggregate the categories that represent the document, unlike the method described in this thesis, which uses the shortest paths towards the main topics, \cite {7062484} used a measure of semantic similarity to find the most related categories. The quality of their method for the classification was only evaluated manually by the authors.

Takeda et al. \cite{takeda2016classification} described a method for the classification of tweets in a system for \gls{ir} in the context of tourism information on social media. They propose a technique for classification that utilizes tree kernels (topic trees) created from categories extracted from  Wikipedia. Although the approach presented in \cite{takeda2016classification} is similar to the method used here, their research did not take full advantage of the rich structure of categories since they only considered paths that are three levels deep. Furthermore, they transformed the graph into a tree by taking into account only one shortest path from each category to the top, while the analysis described in this thesis included the topology of the graph (supporting the decision to keep it as a small-world network). 

Kumar, Rengarajan Annie \cite{kumar2017wikitop} described Wikitop, a method for automatically generating topic trees from the text by performing hierarchical classifications using the Wikipedia Category Structure. The major difference between their approach and the one outlined here is that, while the proposed method assigns flat categories with degrees of pertinence, Wikitop assigns a hierarchical classification (topic trees). Furthermore, the technique has only been tested on a small scale and only with Wikipedia's articles.


