% Início do abstract
Medeiros, Jerry Fernandes. \textbf{TagTheWeb: USING WIKIPEDIA CATEGORIES TO AUTOMATICALLY CATEGORIZE TEXT-BASED RESOURCES ON THE WEB}. UNIRIO, 2018. \pageref{LastPage} pages. Master Thesis. Departamento de Informática Aplicada, UNIRIO.
\vspace{30pt}
\begin{center}
    \textbf{ABSTRACT}
    \vspace{30pt}
\end{center}
Identifying topics associated with a set of documents is a common task for many applications and can be used to improve various tasks involving documents on the Web, such as search, retrieval, recommendation, and clustering. Due to the significant amount of information produced and made available today, it becomes humanly impossible to organize, analyze, and extract the knowledge embedded. Consequently, mechanisms to accomplish such tasks as removing or at least diminishing the need for human intervention has gained importance in the last decades. One of the potential solutions for dealing with the challenge of organizing and retrieving documents is to use automated classification and categorization of Web information. In this research, we propose a generic classification method to automatically categorize any text-based content on the Web according to the collective knowledge of Wikipedia contributors, through the semantic relation between nodes of the Wikipedia Category Graph. Our approach is based on three steps: extracting named entities from text, extracting categories associated with named entities, and finally representing and classifying the document. In order to validate our method, we conducted computational experiments and a study involving users of a crowdsourcing platform. The results show that our approach is capable of correctly categorizing most documents in a way that real users can understand, without the effort of domain experts.

\vspace{15pt}

\textbf{Keywords: Text Classification, Wikipedia, Categories, Category Graph}
% Fim do abstract

