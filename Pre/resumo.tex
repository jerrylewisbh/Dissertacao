% Início do resumo
\begin{center}
    \textbf{RESUMO}
    \vspace{40pt}
\end{center}

A identificação de tópicos associados a um conjunto de documentos é uma tarefa comum para muitas aplicações e pode ser usada para melhorar diversas tarefas envolvendo documentos na Web, tais como a busca, recuperação da informação, recomendação, armazenamento e agrupamento. Devido à quantidade significativa de informações produzidas e disponibilizadas hoje, torna-se humanamente impossível organizar, analisar e extrair o conhecimento incorporado nesses documentos. Consequentemente, mecanismos para realizar tarefas como remover ou pelo menos diminuir a necessidade de intervenção humana ganharam importância nas últimas décadas. Uma das possíveis soluções para lidar com o desafio de organizar e recuperar documentos é usar classificação automatizadas de informações na Web. Nesta pesquisa, propomos um método de classificação genérico para categorizar automaticamente conteúdo baseado em texto na Web de acordo com o conhecimento coletivo dos colaboradores da Wikipedia, por meio da relação semântica entre os nós do Gráfico de Categoria da Wikipédia. Nossa abordagem é baseada em três etapas: extrair entidades nomeadas do texto, extrair categorias associadas a entidades nomeadas e, finalmente, representar e classificar o documento. Para validar nosso método, realizamos experimentos computacionais e um estudo envolvendo usuários de uma plataforma de crowdsourcing. Os resultados mostram que nossa abordagem é capaz de categorizar corretamente a maioria dos documentos de uma maneira que os usuários reais possam entender, sem o esforço dos especialistas em domínio.
\vspace{20pt}

\textbf{palavras-chave:} Classificação de texto, Wikipedia, Categorias, Gráfo de Categorias
% Fim do resumo